%!TEX root = ./template-skripsi.tex
%-------------------------------------------------------------------------------
%                            	BAB IV
%               		KESIMPULAN DAN SARAN
%-------------------------------------------------------------------------------

\chapter{KESIMPULAN DAN SARAN}

\section{Kesimpulan}
Berdasarkan hasil dari implementasi dan pengujian \textit{Automatic Tagging} dengan 
menggunakan algoritma \textit{Bipartite Graph Partition} dan \textit{Two Way Poisson Mixture Model}
, maka diperoleh kesimpulan sebagai berikut. Program yang telah dibuat dapat menghasilkan beberapa rekomendasi \textit{tag}.
Akan tetapi, akurasi yang dihasilkan masih sangat rendah yaitu 37\%. Hal ini dikarenakan banyaknya $K$ yang digunakan dalam program ini
untuk melakukan \textit{Bipartite Graph Partition} hanyalah dua. Selain itu, nilai  $\theta \left(d(i, j) \mid \tilde{\lambda}_{m, i, j}^{(t)}\right)$
yang dihasilkan mempengaruhi nilai $p_{i,m}$.

\section{Saran}
Adapun saran untuk penelitian selanjutnya adalah:
\begin{enumerate} 
	\item Menggunakan jumlah $K$ yang lebih optimal.
	\item Menggunakan dataset yang lebih besar karena tingkat akurasinya 
	akan lebih meningkat dibandingkan dengan dataset yang lebih kecil.
	\item Dapat menggunakan bahasa Indonesia dalam pemilihan artikel atau situs 
	yang digunakan untuk penelitian. 
	\item Mengintegrasikan algoritma ini ke dalam aplikasi \textit{website} agar 
	pengguna mampu menggunakannya lebih mudah.
	\item Melakukan sinkronisasi dari algoritma \textit{Automatic Tagging} ke dalam 
	sistem mesin pencari \textit{Telusuri}.
\end{enumerate}


% Baris ini digunakan untuk membantu dalam melakukan sitasi
% Karena diapit dengan comment, maka baris ini akan diabaikan
% oleh compiler LaTeX.
\begin{comment}
\bibliography{daftar-pustaka}
\end{comment}